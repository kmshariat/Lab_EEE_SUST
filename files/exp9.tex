\section*{Experiment 9: Design and analysis of AND gate, OR gate and NOT gate using BJT}  
\addcontentsline{toc}{section}{Experiment 9: Design and analysis of AND gate, OR gate and NOT gate using BJT}

\subsection*{Learning Objectives:}

\begin{itemize}
    \item Understand the working principle of Bipolar Junction Transistors (BJT)
    \item Implement Logic gates (AND, OR, NOT) using Bipolar Junction Transistors
\end{itemize}

\subsection*{Equipment:}

\begin{itemize}
    \item Breadboard
    \item NPN transistors (e.g \verb|2N2222|)
    \item LEDs (different colors for easier identification)
    \item Resistors 
    \item Push Buttons 
    \item Power Supply 
    \item Jumper Wires
\end{itemize}

\subsection*{Theory}

The use of transistors for the construction of logic gates depends upon their utility as fast switches. When enough voltage is applied to the base-emitter junction , the transistor is turned on i.e. collector voltage with respect to the emitter may be near zero and can be used to construct gates.

\vspace{0.25cm}

\noindent \textbf{AND gate}

\noindent For the AND logic, the two npn transistors are in series and both transistors must be in the conducting state or turned on to drive the output high.

\begin{figure}[H]
    \centering
    \includesvg[width=0.65\linewidth]{img/and_gate.svg}
    \caption{AND gate implementation using BJT}
    \label{fig:and_gate}
\end{figure}

\vspace{0.25cm}

\noindent \textbf{OR gate}

\noindent For the OR logic, the two npn transistors are parallel connected. The output is driven high if either of the transistors is conducting.

\begin{figure}[H]
    \centering
    \includesvg[width=0.65\linewidth]{img/or_gate.svg}
    \caption{OR gate implementation using BJT}
    \label{fig:or_gate}
\end{figure}

\vspace{0.25cm}

\noindent \textbf{NOT gate}

\noindent A transistor with a collector resistor can serve as an inverter

\begin{figure}[H]
    \centering
    \includesvg[width=0.65\linewidth]{img/not_gate.svg}
    \caption{NOT gate implementation using BJT}
    \label{fig:not_gate}
\end{figure}

\subsection*{Procedure}
\begin{itemize}
    \item Built circuits according to fig-\ref{fig:and_gate}, fig-\ref{fig:or_gate}, fig-\ref{fig:not_gate}
    \item Give 5V DC power supplies ( two power supplies for AND and OR; one for NOR) across the base- emitter terminal of the BJT
    \item Turn on/off the power supplied according to the truth tables. Consider 5V as binary 1 and 0V as binary 0.
    \item Measure the output
\end{itemize}

\subsection*{Tabulation}

\noindent \textbf{AND Gate}
\begin{table}[h]
    \centering
    \begin{tabular}{c|c|c}
        \hline
        A & B & Output \\ \hline \hline
        0   & 0  &   \\ \hline
        0   & 1   &    \\ \hline
        1  & 0   &    \\ \hline
        1  & 1  &   \\ \hline
    \end{tabular}
    \label{tab:and_gate}
\end{table}

\noindent \textbf{OR Gate}
\begin{table}[h]
    \centering
    \begin{tabular}{c|c|c}
        \hline
        A & B & Output \\ \hline \hline
        0   & 0  &   \\ \hline
        0   & 1   &    \\ \hline
        1  & 0   &    \\ \hline
        1  & 1  &   \\ \hline
    \end{tabular}
    \label{tab:or_gate}
\end{table}

\noindent \textbf{NOT Gate}
\begin{table}[h]
    \centering
    \begin{tabular}{c|c}
        \hline
        A  & Output \\ \hline \hline
        0   &      \\ \hline
        1  &        \\ \hline

    \end{tabular}
    \label{tab:not_gate}
\end{table}

\subsection*{Conclusion}
This experiment demonstrated the practical application of transistors in designing basic TTL logic gates. By constructing and testing AND, OR, and NOT gates, students gained hands-on experience in understanding transistor-based digital logic circuits.

\newpage