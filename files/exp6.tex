\section*{Experiment 6: Verification of Maximum Power Transfer Theorem}  
\addcontentsline{toc}{section}{Experiment 6: Verification of Maximum Power Transfer Theorem}

\subsection*{Learning Objectives:}
\begin{itemize}
    \item Understand the concept of maximum power transfer theorem.
    \item Verify the theorem through experimentation with a DC circuit.
\end{itemize}

\subsection*{Equipment:}
\begin{itemize}
    \item DC Power Supply (adjustable voltage)
    \item Breadboard
    \item Resistors (various values - $100 \Omega$, $220 \Omega$, $330 \Omega$, $1k\Omega$)
    \item Multimeter
\end{itemize}

\subsection*{Procedure:}

\begin{enumerate}
    \item \textbf{Circuit Construction:} Build the following circuit on the breadboard:

    \begin{figure}[H]
        \begin{center}
            \begin{tikzpicture}
	% Paths, nodes and wires:
	\draw (1.25, 6) to[american voltage source, l_={$V$}, label distance=0.02cm] (1.25, 2);
	\draw (1.25, 6) to[american resistor, l={$R_1$}, label distance=0.02cm] (4.25, 6);
	\draw (4.25, 6) to[american resistor, l={$R_3$}, label distance=0.02cm] (7.25, 6);
	\draw (4.25, 6) to[american resistor, l={$R_2$}, label distance=0.02cm] (4.25, 2);
	\draw (7.25, 6) to[ammeter, l={$I_L$}, label distance=0.02cm] (7.25, 4);
	\draw (7.25, 4) to[american resistor, l={$R_L$}, label distance=0.02cm] (7.25, 2);
	\draw (1.25, 2) -- (7.25, 2);
	\draw (9, 6) to[american voltage source, l={$V_{TH}$}, label distance=0.02cm] (9, 2);
	\draw (9, 6) to[american resistor, l={$R_i$}, label distance=0.02cm] (12, 6);
	\draw (12, 6) to[ammeter, l={$I_L$}, label distance=0.02cm] (12, 4);
	\draw (12, 4) to[american resistor, l={$R_L$}, label distance=0.02cm] (12, 2);
	\draw (9, 2) -- (12, 2);
\end{tikzpicture}
        \end{center}
    \end{figure}

    \begin{itemize}
        \item The circuit consists of a DC power supply connected as shown in the figure above. The variable load resistor ($R_L$) and fixed resistors ($R_1$, $R_2$ and $R_3$). The internal resistance ($R_i$) and $V_th$ represent Thevenin’s equivalent resistance and voltage supply respectively.
    \end{itemize}

    \item \textbf{Voltage and Current Measurement:} Set the multimeter to measure voltage (DC Volts) and current (DC Amps).

    \item \textbf{Variable Load:}

    \begin{itemize}
        \item Adjust the voltage of the DC power supply to a fixed value (e.g., $5V$).
        \item Measure the voltage across the load resistor ($V_L$) and the current flowing through the load resistor ($I_L$) for various values of the load resistance ($R_L$) by adjusting the variable resistor. Record these values in a table.
    \end{itemize}

\end{enumerate}

\subsection*{Data and Calculations:}
\begin{itemize}
    \item For each set of readings, calculate the power delivered to the load resistor ($P_L$) using the formula $P_L = V_L \times I_L$.
    \item Plot a graph of the calculated load power ($P_L$) versus the load resistance ($R_L$).
\end{itemize}

\begin{table}[H]
    \centering
    \begin{tabular}{m{1cm}|m{2cm}|m{2.5cm}|m{2.5cm}|m{2cm}|m{2cm}}
        \hline
        Serial no & Load resistance, $R_L$ & Load Current, $I_L$ [observed] & Load Current, $I_L$ [calculated] & Power, $P$ [observed] & Power, $P$ [calculated] \\
        \hline
          &   &   &   &  & \\
        \hline
          &   &   &   &  & \\
        \hline

    \end{tabular}
    \label{tab:placeholder_label}
\end{table}

\subsection*{Analysis:}
\begin{itemize}
    \item Observe the shape of the power curve.
    \item Identify the point on the graph where the load power reaches its maximum value.
    \item At this point, calculate the value of the load resistance ($RL_{max}$).
\end{itemize}

\subsection*{Theoretical Verification:}
If the internal resistance of the DC power supply ($R_i$) is known, calculate the value of the load resistance for maximum power transfer using the formula: $RL_{max} = R_i$. The formula of maximum power transfer, $P_{max} = (V_{th})^2 / (4 RL_{max})$

\subsection*{Conclusion:}
By observing the graph and comparing the measured maximum power transfer resistance ($RL_{max}$) with the calculated value (if $R_i$ is known), verify the principle of the maximum power transfer theorem.

\newpage
