\section*{Experiment 4: Verification of Kirchhoff's Current Law (KCL) and Kirchhoff's Voltage Law (KVL)}  
\addcontentsline{toc}{section}{Experiment 4: Verification of Kirchhoff's Current Law (KCL) and Kirchhoff's Voltage Law (KVL)}

\subsection*{Learning Objectives:}
\begin{itemize}
    \item Understand Kirchhoff's Current Law (KCL) and Kirchhoff's Voltage Law (KVL).
    \item Verify KCL and KVL through experimentation in a DC circuit.
\end{itemize}

\subsection*{Equipment:}
\begin{itemize}
    \item DC Power Supply
    \item Resistors (various values - $100 \Omega$, $220 \Omega$, $330\Omega$)
    \item Multimeter
    \item Jumper Wires
\end{itemize}

\subsection*{Procedure:}

\noindent \textbf{Part A: Verification of KCL}
\begin{enumerate}

    \item \textbf{Circuit Construction:} Build the following circuit on the breadboard:
    \begin{itemize}
        \item The circuit consists of a DC power supply connected to a point (node) where three resistors (R1, R2, R3) are connected together.
    \end{itemize}
    
    \item \textbf{Current Measurement:} Set the multimeter to measure current (DC Amps).

    \begin{figure}[H]
        \centering
        \includesvg[width=0.6\linewidth]{img/kcl.svg}
        \caption{Verification of KCL}
        \label{fig:enter-label}
    \end{figure}
    
    \item \textbf{Measuring Branch Currents:} Measure the current flowing through each branch (individual resistors) leading away from the junction ($I_1$ and $I_2$). Record these values in a table.
    
    \item \textbf{Measuring Current at the Junction:} Measure the total current entering the junction ($I_{total}$) by connecting the multimeter in series with the power supply feeding the circuit. Record this value.

    \item \textbf{KCL Verification:} Compare the sum of the branch currents ($I_1 + I_2$) with the total current entering the junction ($I_{total}$). According to KCL, these values should be equal.

    \item Compare the value with the theoretical results.
\end{enumerate}

\noindent \textbf{Part B: Verification of KVL} 

\begin{enumerate}
    \item \textbf{Circuit Modification:} Modify the circuit from Part A like the figure.

    \item \textbf{Voltage Measurement: }Set the multimeter to measure voltage (DC Volts).

    \begin{figure}[H]
        \centering
        \includesvg[width=0.65\linewidth]{img/kvl.svg}
        \caption{Verification of KVL}
        \label{fig:kvl}
    \end{figure}

    \item \textbf{Measuring Individual Voltages:} Measure the voltage drop across each resistor in the closed loop of the circuit ($V_{R1}$, $V_{R2}$, $V_{R3}$). Record these values in a table.

    \item \textbf{KVL Verification:} Add the voltage drops across all the resistors in the closed loop ($V_{R1} + V_{R3} - V$) and ($V_{R2} - V_{R3}$). According to KVL, the algebraic sum of these voltages should be equal to zero.

    \item Compare the value with the theoretical results.

\end{enumerate}

\subsection*{Observation Table:}

\noindent \textbf{KCL}

\begin{table}[h]
    \centering
    \begin{tabular}{m{1.5 cm}|m{1.5 cm}|m{1.5 cm}|m{3.5 cm}|m{3.5 cm}}
        \hline
        Input Voltage, V & $I_1$ & $I_2$ & Total Current, $I_{total}$ [measured value] & Total Current, $I_{total}$ [theoretical value] \\
        \hline
         &   &  &  &  \\
        \hline

    \end{tabular}

    \label{tab:kcl}
\end{table}

\noindent \textbf{KVL}

\begin{table}[h]
    \centering
    \begin{tabular}{m{1.5 cm}|m{1.5 cm}|m{1.25 cm}|m{1.25 cm}|m{1.25 cm}|m{3 cm}|m{3 cm}}
        \hline
        Loop no. & Input Voltage, V & $V_{R1}$ & $V_{R2}$ & $V_{R3}$ & Algebraic Sum of Voltages [measured value] & Algebraic Sum of Voltages [theoretical value] \\
        \hline
         &   &  &  &  & \\
        \hline

    \end{tabular}
    \label{tab:kvl}
\end{table}

\subsection*{Data and Calculations:}

\begin{itemize}
    \item In a table, record the measured values of branch currents, total current, and individual voltage drops.
    \item Calculate the expected total current based on KCL (sum of branch currents).
    \item Calculate the expected sum of voltage drops around the closed loop based on KVL (sum of individual voltage drops).
\end{itemize}

\subsection*{Conclusion:}
By comparing the measured and theoretical values, verify the validity of Kirchhoff's Current Law and Kirchhoff's Voltage Law for DC circuits.

\newpage 